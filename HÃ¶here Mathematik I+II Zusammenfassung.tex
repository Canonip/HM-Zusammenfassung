\documentclass[a4paper,portrait]{scrartcl}
\author{Andreas Mai}
\title{HM I + II Zusammenfassung KIT}
\usepackage[utf8]{inputenc}
\usepackage[T1]{fontenc}
\usepackage{lmodern}
\usepackage[german]{babel}

\begin{document}

\maketitle
\begin{center}
\textbf{HM Klausur am 30.08.2016} \\
\textbf{08:00 - 10:00 HM I} \\
\textbf{11:00 - 13:00 HM II} 
\end{center}

\begin{center}
\textit{Kein Anspruch auf Vollständigkeit ;)}
\end{center}\clearpage
\tableofcontents
\clearpage
\setcounter{page}{1}
\section{Konvergenz}
\begin{itemize}
  \item Eine Folge (oder Reihe) ist konvergent, wenn sie gegen einen bestimmten Wert konvergiert.
  \item Sie ist bestimmt divergent, wenn sie gegen $ \pm  \infty $ läuft
  \item Sie ist unbestimmt divergent, wenn sich keine Aussage machen lässt \\ (bsp: 1 und -1 abwechselnd)
\end{itemize}

\subsection{Vorgehen}
\subsubsection{Grenzwert bestimmen}
Grad der Funktion:
\begin{itemize}
  \item $Zählergrad < Nennergrad \Rightarrow  S_{n} \rightarrow 0$ \\
  Beispiel: $S_{n} = \frac{n}{n^2+4} \Rightarrow S_{n} \rightarrow 0$
  \item $Zählergrad = Nennergrad \Rightarrow  S_{n} \rightarrow Bruch$ \\
    Beispiel: $S_{n} = \frac{3n+4}{5n+96} \Rightarrow S_{n} \rightarrow \frac{3}{5}$
  \item $Zählergrad > Nennergrad \Rightarrow  S_{n} \rightarrow \infty \Rightarrow $ bestimmt divergent \\
    Beispiel: $S_{n} = \frac{n^6-7}{n^2+4} \Rightarrow S_{n} \rightarrow \infty$
\end{itemize}

\end{document}
